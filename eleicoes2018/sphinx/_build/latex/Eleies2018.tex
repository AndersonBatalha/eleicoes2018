%% Generated by Sphinx.
\def\sphinxdocclass{report}
\documentclass[letterpaper,10pt,portuges]{sphinxmanual}
\ifdefined\pdfpxdimen
   \let\sphinxpxdimen\pdfpxdimen\else\newdimen\sphinxpxdimen
\fi \sphinxpxdimen=.75bp\relax

\PassOptionsToPackage{warn}{textcomp}
\usepackage[utf8]{inputenc}
\ifdefined\DeclareUnicodeCharacter
% support both utf8 and utf8x syntaxes
\edef\sphinxdqmaybe{\ifdefined\DeclareUnicodeCharacterAsOptional\string"\fi}
  \DeclareUnicodeCharacter{\sphinxdqmaybe00A0}{\nobreakspace}
  \DeclareUnicodeCharacter{\sphinxdqmaybe2500}{\sphinxunichar{2500}}
  \DeclareUnicodeCharacter{\sphinxdqmaybe2502}{\sphinxunichar{2502}}
  \DeclareUnicodeCharacter{\sphinxdqmaybe2514}{\sphinxunichar{2514}}
  \DeclareUnicodeCharacter{\sphinxdqmaybe251C}{\sphinxunichar{251C}}
  \DeclareUnicodeCharacter{\sphinxdqmaybe2572}{\textbackslash}
\fi
\usepackage{cmap}
\usepackage[T1]{fontenc}
\usepackage{amsmath,amssymb,amstext}
\usepackage{babel}
\usepackage{times}
\usepackage[Sonny]{fncychap}
\ChNameVar{\Large\normalfont\sffamily}
\ChTitleVar{\Large\normalfont\sffamily}
\usepackage{sphinx}

\fvset{fontsize=\small}
\usepackage{geometry}

% Include hyperref last.
\usepackage{hyperref}
% Fix anchor placement for figures with captions.
\usepackage{hypcap}% it must be loaded after hyperref.
% Set up styles of URL: it should be placed after hyperref.
\urlstyle{same}
\addto\captionsportuges{\renewcommand{\contentsname}{Contents:}}

\addto\captionsportuges{\renewcommand{\figurename}{Fig.}}
\addto\captionsportuges{\renewcommand{\tablename}{Table}}
\addto\captionsportuges{\renewcommand{\literalblockname}{Listing}}

\addto\captionsportuges{\renewcommand{\literalblockcontinuedname}{continuação da página anterior}}
\addto\captionsportuges{\renewcommand{\literalblockcontinuesname}{continues on next page}}
\addto\captionsportuges{\renewcommand{\sphinxnonalphabeticalgroupname}{Non-alphabetical}}
\addto\captionsportuges{\renewcommand{\sphinxsymbolsname}{Símbolos}}
\addto\captionsportuges{\renewcommand{\sphinxnumbersname}{Numbers}}

\addto\extrasportuges{\def\pageautorefname{page}}

\setcounter{tocdepth}{1}



\title{Eleições 2018 Documentation}
\date{01 dez, 2018}
\release{1.0.0}
\author{Anderson Pontes Batalha}
\newcommand{\sphinxlogo}{\vbox{}}
\renewcommand{\releasename}{Versão}
\makeindex
\begin{document}

\ifdefined\shorthandoff
  \ifnum\catcode`\=\string=\active\shorthandoff{=}\fi
  \ifnum\catcode`\"=\active\shorthandoff{"}\fi
\fi

\pagestyle{empty}
\maketitle
\pagestyle{plain}
\sphinxtableofcontents
\pagestyle{normal}
\phantomsection\label{\detokenize{index::doc}}



\chapter{Introdução}
\label{\detokenize{introducao:introducao}}\label{\detokenize{introducao::doc}}
O aplicativo deverá, com base nos dados disponibilizados pelo Tribunal Superior Eleitoral, exibir os dados dos candidatos nas eleições de 2018, para os cargos de presidente e governador.

A cada eleição, o TSE disponibiliza um compilado de diversas informações referentes a eleição (candidatos, partidos, prestação de contas, resultados, entre outras coisas). Esses arquivos geralmente estão disponíveis em formatos .txt ou .csv, e estão abertos a qualquer pessoa que queira analisar essas informaçoes.

Este projeto faz parte da disciplina de Desenvolvimento Web do curso de Sistemas de Informação, feito por Anderson Pontes Batalha.

O aplicativo web foi desenvolvido com o framework web Django, para mais detalhes sobre o framework, acesse \sphinxurl{http://djangoproject.com}.


\chapter{Baixando o projeto}
\label{\detokenize{clonar_repositorio_git:baixando-o-projeto}}\label{\detokenize{clonar_repositorio_git::doc}}
Para baixar o projeto em sua maquina, e necessario ter o git instalado. A instalaçao pode ser feita com:

\fvset{hllines={, ,}}%
\begin{sphinxVerbatim}[commandchars=\\\{\}]
\PYGZdl{} sudo apt install git
\end{sphinxVerbatim}

O repositorio esta disponivel em \sphinxurl{https://github.com/AndersonBatalha/eleicoes2018.git}. Basta copiar o link, e ainda no terminal, executar:

\fvset{hllines={, ,}}%
\begin{sphinxVerbatim}[commandchars=\\\{\}]
\PYGZdl{} git clone https://github.com/AndersonBatalha/eleicoes2018.git
\end{sphinxVerbatim}


\chapter{Instalação das dependências}
\label{\detokenize{instalacao_dependencias:instalacao-das-dependencias}}\label{\detokenize{instalacao_dependencias::doc}}
Para executar o projeto e necessário ter instalado em sua máquina:
\begin{itemize}
\item {} 
Python 3.6.7

\item {} 
Django 1.9.10

\item {} 
virtualenv

\item {} 
pip (gerenciador de pacotes do Python)

\end{itemize}

Supondo que voce utilize Linux como sistema operacional, basta instalar pelo terminal:

\fvset{hllines={, ,}}%
\begin{sphinxVerbatim}[commandchars=\\\{\}]
\PYGZdl{} sudo apt install python3.6 virtualenv virtualenvwrapper python\PYGZhy{}pip
\end{sphinxVerbatim}


\section{Criando o ambiente virtual}
\label{\detokenize{instalacao_dependencias:criando-o-ambiente-virtual}}\begin{quote}

Um ambiente virtual se caracteriza por isolar as aplicações instaladas para um determinado projeto, evitando conflitos de versões entre pacotes iguais que pertençam a projetos diferentes.
\end{quote}
\begin{enumerate}
\def\theenumi{\arabic{enumi}}
\def\labelenumi{\theenumi .}
\makeatletter\def\p@enumii{\p@enumi \theenumi .}\makeatother
\item {} 
Abra o terminal e execute:

\end{enumerate}

\fvset{hllines={, ,}}%
\begin{sphinxVerbatim}[commandchars=\\\{\}]
\PYGZdl{} mkvirtualenv \PYGZhy{}p \PYG{l+s+sb}{{}`}which python3\PYG{l+s+sb}{{}`} env
\end{sphinxVerbatim}
\begin{enumerate}
\def\theenumi{\arabic{enumi}}
\def\labelenumi{\theenumi .}
\makeatletter\def\p@enumii{\p@enumi \theenumi .}\makeatother
\setcounter{enumi}{1}
\item {} 
Para instalar as dependências, entre na pasta do projeto e localize o arquivo requirements.txt. Esse arquivo contem os pacotes necessários para a execução do projeto.

\end{enumerate}

\fvset{hllines={, ,}}%
\begin{sphinxVerbatim}[commandchars=\\\{\}]
\PYGZdl{} pip install \PYGZhy{}r requirements.txt
\end{sphinxVerbatim}

E assim o projeto está pronto para ser executado.


\chapter{Script para popular o banco de dados}
\label{\detokenize{popular_banco:script-para-popular-o-banco-de-dados}}\label{\detokenize{popular_banco::doc}}
Na pasta do projeto, localize o arquivo “popular\_candidatos.py”. Esse arquivo é responsável por extrair os dados dos arquivos csv e popular as tabelas do banco de dados. Ainda no terminal, execute:

\fvset{hllines={, ,}}%
\begin{sphinxVerbatim}[commandchars=\\\{\}]
\PYGZdl{} python populate\PYGZus{}candidatos.py
\end{sphinxVerbatim}


\chapter{Executando o projeto}
\label{\detokenize{executar_projeto:executando-o-projeto}}\label{\detokenize{executar_projeto::doc}}\begin{enumerate}
\def\theenumi{\arabic{enumi}}
\def\labelenumi{\theenumi .}
\makeatletter\def\p@enumii{\p@enumi \theenumi .}\makeatother
\item {} 
Após acessar a pasta onde está localizado o projeto e execute no terminal:

\end{enumerate}

\fvset{hllines={, ,}}%
\begin{sphinxVerbatim}[commandchars=\\\{\}]
\PYGZdl{} python manage.py runserver
\end{sphinxVerbatim}
\begin{enumerate}
\def\theenumi{\arabic{enumi}}
\def\labelenumi{\theenumi .}
\makeatletter\def\p@enumii{\p@enumi \theenumi .}\makeatother
\setcounter{enumi}{1}
\item {} 
Em seguida, abra seu navegador preferido, e digite a URL:

\end{enumerate}

\sphinxurl{http://localhost:8000}


\chapter{Indices and tables}
\label{\detokenize{index:indices-and-tables}}\begin{itemize}
\item {} 
\DUrole{xref,std,std-ref}{genindex}

\item {} 
\DUrole{xref,std,std-ref}{modindex}

\item {} 
\DUrole{xref,std,std-ref}{search}

\end{itemize}



\renewcommand{\indexname}{Índice}
\printindex
\end{document}